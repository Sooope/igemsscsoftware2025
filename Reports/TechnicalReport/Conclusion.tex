\section{Conclusion}
Our proposed eczema diagnosis platform can help address with the global burden on eczema diagnosis by using deep learning models to automatically finish diagnosis works on a user-friendly cross-platform application. The report detailedly analyzed the relevant technical approaches systematically across three important stages: data, model and software.

In terms of data, the two main tasks are data collection and data augmentation. The data collection work would mainly focus on collecting data from internet. Including well-labeled dataset built by researchers, data scientists and enterprises and data from websites through web scraping. The main data we will need for our platform are images of eczema and text data about eczema diagnosis. Then, for data augmentation, the main task is cleaning and augmenting the data to get better dataset. There are plentiful method toward data augmentation, by using correct method, the quality of dataset can be improved, thus resulting in better model performance.

For model, we would have in total three models, including an initial diagnosis model that classify whether the user get eczema or not, classification model to classify the eczema subtypes and a personal suggestion model to provide personalized suggestion to users based on their situation. For initial diagnosis model, it is possible to have both text and image input, former will be processed by CNNs, latter will need NLP models. The classification model could be achieved by either CNNs or object detection algorithms. At last, the suggestion model could be done without deep learning algorithms, but for better personalization performance, we could add NLP to gain personalized suggestion. Apart from model design, the training of model would need building of distributed training clusters with data parallel training methodology. The evaluation part would involve choosing suitable benchmarks and conduct evaluation toward models.

The software development stage would consist the deployment of model, development of software and the UI/UX design. The deployment of model would involve technique to compress the model size and computational complexity, including methodologies like pruning, quantization and knowledge distillation. The development of cross-platform application would need a cross-platform application SDK, which is probably Flutter, that allows using one codebase to construct application run on multiple platforms. The UI/UX part should obey the principle of providing intuitive instruction and clear information.

There are possible future enhancement including the dermoscopy based diagnosis, severity assessment, patient-doctor community and adding other dermatological disease. 