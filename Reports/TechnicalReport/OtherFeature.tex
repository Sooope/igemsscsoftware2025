\section{Other possible features}
Apart from the proposed features mentioned above, there are features that may be achieved in future as well. This section proposed some additional features that also have potential to be implemented into our platform.

Firstly, our recent proposed platform can only identify macro-images that take by the camera of the affected area, and all the diagnosis are based on the image and the description from users. However, this is different from the actual workflow in clinic dermatological diagnosis, which instead of focusing on the appearance of affected area only, clinic dermatological would conduct examination like dermoscopy to have further check. Therefore, it is possible to made diagnosis models based on dermoscopy inputs. The main challenge need to come across is the dataset, a model that diagnosis based on dermoscopy input would require massive amount of dermoscopy data, even advance unsupervised models can conduct classification work without labeled data, the difficulty to collect large amount of personal medical data is still high. If with enough resources to collect data, it is still possible to make a model aid doctors' diagnosis in the future, which need at least hundreds images per class.

Secondly, our models can only work on classification works like classifying the eczema subtypes, assessment on the eczema severity is not available in our platform. While an identification model on users' eczema severity is not impossible, there are works like \cite{ATTAR2023100213}\cite{10.1007/978-3-030-59861-7_23} use machine learning approaches built models that can perform severity assessment tasks, but similarly, the limited data resources make it hard, the general solution toward the problem is still on predicting severity based on the area of affected area. However, if the identification toward eczema severity can be achieved, it will not only provide extra information to users, but also provide more information for the suggestion model. Therefore, the severity identification would be a possible feature to be added later.

Apart from above improvement on models, we can work on adding other feature to the platform. For instance, a patient doctor community. A community that provide a platform to patients and doctors to communicate, including the communication between patients or between doctors and patients. With a community, patients can communicate and exchange experience on eczema treating, remote diagnosis with professional dermatologists is also possible, the result of models can thus aid the diagnosis. However, such a platform would cost extra resources on sever and development, but it is still a considerable feature to add on in the future.

Furthermore, it is also possible to add on other dermatological diseases to further develop the platform. Thus, the platform would become not only an eczema platform, but also an overall dermatology platform. Similarly, the improvement has to be done with massive amount of data and training resources to build up a general dermatology model. With sufficient resources, such an improvement would benefit larger range of people who suffer from other kind of dermatological diseases.

In conclusion, there plentiful features can be added on in the future. The main challenge toward the implement is still the data. Establish of more specific or general dermatology platform would need massive invest in data collection and algorithm optimization.