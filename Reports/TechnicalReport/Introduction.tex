\section{Introduction}

Eczema, also known as atopic dermatitis, is a very common chronic, inflammatory skin disease\cite{deckers2012investigating}. It has large global burden\cite{laughter2021global}, and affected about 20\% of children and 10\% of adult in developed countries\cite{odhiambo2009global}. 

Traditionally, the diagnosis of eczema heavily relies on manual examination by dermatologists, and the diagnosis process involves classification between different eczema subtype. An accurate classification and diagnosis is crucial in clinical treatment, since different subtype of eczema would need different method for treatment. The accuracy and constancy of diagnosis therefore became important in clinic\cite{ADD}. Furthermore, eczema may have similar appearance from other common skin disease, the similarity cause potential misdiagnosis in clinic if doctors may not have received sufficient professional education in dermatology. Research shows a high misdiagnosis percentage in clinic when dealing with skin diseases.

Over last decades, with the development of machine learning algorithm, automatic eczema diagnosis became possible. There are works that focus on making model for eczema diagnosis, including classification of eczema and other skin disease(e.g. acne), automatic severity evaluation and classification between eczema subtype, among the recent works, deep learning algorithm is the most popular approach used\cite{huang2024remote}. However, a robust model that allows an accessible deployment into software have not been achieved.

Moreover, current research only focuses on one specific task in eczema diagnosis, for instance, classification of eczema subtypes. A general platform that include the whole workflow of eczema diagnosis has not been investigated, which cause the clinical application of machine learning and computer vision technologies in eczema diagnosis impossible.

To fill with the lack of robust model in this field and enhance the application of machine learning into practice, we proposed a general eczema platform that consist diagnosis, personalized suggestion and relevant information. 

We first break down the problem into different stages, which will be further explained in section \ref{problem}. Generally, the problem can be seen as 3 stages, data collection and augmentation, model building, and software development. In each stage, the problem can be further divided into several specific tasks, each task may have multiple solution.

To achieve such a platform, we need to train robust models that can classify eczema subtypes, diagnosis eczema and provide personalized information. The main challenge would come from data, as the fundamental part of machine learning, high-quality data on eczema is insufficient, and it has become the main bottleneck in the prior works\cite{huang2024remote}. To cope with that, the augmentation of data is crucial.

Model building under limiting data is important as well, a transfer learning might be nessccesary to obtain a considerable accuracy. There are also multiple model architecture available, including convolution neural networks, object detection and transformer. The training of model would be challenging as well if we need to build a large model, training on multiple GPUs seems to be nessccesary.

In software development, to make the platform more accessible, we would like to create a cross-platform software, including access from PC, iOS and Android. This might come up with the problem of model efficiency that the performance of mobile device may not be able to support running of large model, it might require technic like model distillation to ensure the performance of model in edges.

\section{Related works}

There are several works on using machine learning algorithms for dermatological diagnosis. This section would review some works done before.

In terms of classifying eczema with other dermatological disease, \cite{juwairi2023efficientnet} use transfer learning on EfficientNet to classify between eczema and acne and deploy the model into Android with TensorFlow Lite, they achieved more than 90\% accuracy with less than 0.04 seconds respond speed. Similarly, \cite{pathriclassification} use EfficientNet transfer learning to achieve satisfying result. \cite{mehboob2025deep} use deep learning convolutional neural networks on HAM10000 dataset achieved more than 90\% accuracy.

Other work like \cite{thomsen2020deep} used VGG-16 model trained on more than 15,000 dermatological image data on binary classification problem, achieve similar or even better result compare to clinical doctors. \cite{zhou2017multi} use CNNs on a six class classification problem in dermatological images, proven to have better performance than 149 professional dermatologists.

In the aspect of classifying eczema subtypes, \cite{junayed2020eczemanet} create a new CNNs architecture EczemaNet achieved more than 95\% accuracy on classifying five eczema subtypes. 